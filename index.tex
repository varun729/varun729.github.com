% LaTeX file for resume 
% This file uses the resume document class (res.cls)
\documentclass[10pt]{article} 


%%%%% PACKAGES
%
%
%\usepackage{layout}
\usepackage{helvet}
\usepackage{hyperref}
\usepackage{enumitem}
\usepackage{ifthen}


%%%%% PAGE LAYOUT
%
%
\hoffset -30pt
\voffset -30pt
\oddsidemargin 0pt
\topmargin 0pt
\headheight 0pt
\headsep 0pt
\textheight 720pt
\textwidth 510pt
\marginparsep 0pt
\marginparwidth 0pt
\footskip 0pt

\pagestyle{empty}


%%%%% COMMANDS
%
%
\newenvironment{project}[3][]{
	\item \textbf{#2} \ifthenelse{\equal{#1}{}}{}{[#1]}\\
	Guided by \textbf{#3}
	\begin{itemize}[itemsep=0pt]
}{
\end{itemize}
}

\newenvironment{selfproject}[2][]{
	\item \textbf{#2} \ifthenelse{\equal{#1}{}}{}{[#1]}
	\begin{itemize}[itemsep=0pt]
}{
\end{itemize}
}

\newenvironment{projectlist}{
\begin{enumerate}[itemsep=0pt]
}{
\end{enumerate}
}

\newenvironment{techlang}[1]{
\textbf{#1}
\begin{itemize}[itemsep=0pt]
}{
\end{itemize}
}

\newcommand{\sectionhead}[1]{%
\section*{\underline{#1}}
}


%%%%% BEGIN ACTUAL DOCUMENT
%
%
\begin{document} 


%%%%% HEADER
%
%
\begin{center}
\textbf{\textsc{\Large Varun Agrawal}}\\
Application User ID (Email) : varun729@gmail.com\\
Phone : +91 9620660970\\
\hrulefill
\end{center}


%%%%% EDUCATION
%
%
\sectionhead{Education}
\begin{tabular*}{\textwidth}{@{\extracolsep{\fill}} |l|l|l|l|}
\hline
\textbf{Degree/Certificate}  & \textbf{Year} & \textbf{Institute/School}, \textbf{City} & \textbf{CPI/\%} \\
\hline
B. Tech., Electrical Engineering  & 2006-2010 & Indian Institute of Technology Kanpur & 8.3/10\\
\hline
Class XII, CBSE  & 2005-2006 & Delhi Public School, Vindhyanagar & 90\% \\
\hline
Class X, CBSE  & 2003-2004 & Delhi Public School, Vindhyanagar & 93\%  \\
\hline
\end{tabular*}


%%%%% OBJECTIVE
%
%


%%%%% EXPERIENCE
%
%
\sectionhead{Work Experience}
\textbf{Associate(Software)}[August 2010 - Present] at \textbf{Strand Life Sciences}, Bangalore.
	\begin{itemize}[itemsep=0pt]
	\item Member of the software development team for GeneSpring.
	\item Worked on research projects.
	\end{itemize}


%%%%% PROJECTS
\sectionhead{Key Academic Projects}
\begin{projectlist}
\begin{project}[October 2011 - Present]{Indexing of the Reference Genome}{Dr. Ramesh Hariharan, Founder, Strand Life Sciences}
\item Burrows-Wheeler Transform of the reference is used as the index. The indexing was done after removing all the N's(unknown characters) from the reference. The N's are stored separately, and are used only while reconstructing the reference from the BWT.
\item The BWT index is used for aligning reads(sub-sequence of the genome) on the reference.
\item Further work involves alignment of spliced reads, parts of which may align to different regions in the reference genome.
\end{project}
%
\begin{project}[August 2011 - Present]{Designing ontology for Interaction Database}{Dr. Vamsi Veeramachaneni, Vice President, Strand Life Sciences}
\item Designed an ontology for biological interactions. The data is stored in a Resource Description Framework(RDF) model defined by the ontology.
\item In comparison to querying the relational database, it is now very easy to make logical queries to the data to see if their are any anomalies. An example is to look for two entities which are up-regulating each other.
\end{project}
%
\begin{project}[September 2009 - April 2010]{B.Tech Project : Simulation of radars}{Prof. A.R. Harish, Associate Professor, EE, IIT Kanpur}
\item Modeled a glacier surface, and simulated microwaves to estimate the height of the glacier.
\item The irregular surface model of the glacier simulated the surface scattering from an actual glacier very closely.
\item Finite Difference Time Domain(FDTD) technique was used to solve the Maxwell's equations.
\item This simulation tool can be useful in studying the results of the ongoing research in finding glacier heights in Greenland and other icy regions.
\end{project}
%oled
\begin{project}[May 2008 - July 2008]{Summer Project : Organic LED Simulation}{Prof. Deepak Gupta, Professor, Material Science Engineering, IIT Kanpur}
%[Aug-December, 2008]
\item Solved partial differential equations numerically, to find the current flowing in the LED under steady state. Using the simulation, current-voltage(I-V) graphs of several organic materials was studied.
\end{project}
\end{projectlist}



%%%%% TECHNICAL
%
%
\sectionhead{Relevant Courses}
\begin{techlang}{Courses}
\item Linear Algebra, Probability and Statistics, Numerical Computation in Engineering, Linear Programming, Digital Electronics, Signal Processing, Algorithms(Summer course)
\end{techlang}
\begin{techlang}{Programming Languages and Tools}
\item Java, C++, Python, Haskell, \LaTeX{}, MATLAB, SQL, PHP
\item Familiar with Bash scripting, Subversion, Git
\end{techlang}


%%%%% INTERESTS
%
%
\sectionhead{Interests}
\begin{projectlist}
\begin{selfproject}[March 2011 - May 2011]{MyCycle : Android application to record cycling/jogging tracks}
\item Records a jogging or cycling track and shares it on a social network for active people,  Dailymile.
\item Uses the GPS device available in the phone, to get the coordinates. GPS is used economically so that the phone battery is not drained.
\item The application has been downloaded more than 500 times on the Android Market since June 2011.
\end{selfproject}
%
\begin{selfproject}[June 2011 - July 2011]{droidfeedback : Feedback for Android applications}
\item Developed a website and a plugin for Android applications. Any developer can link their Android app to the website, such that feedback questions can be uploaded on the website, and will be automatically visible on the users' phones.
\item This system was used in the application MyCycle, and resulted in excellent response. 8 out of 10 people liked the idea of feedback from inside the application (In a response to one of the questions in the feedback form).
\item The website, \url{http://www.droidfeedback.com/}, is public and is hosted on Amazon EC2 cloud.
\item The plugin is available at \url{https://github.com/varun729/droidfeedback}.
\end{selfproject}
\end{projectlist}


%%%%% ACHIEVEMENTS
%
%
\sectionhead{Achievements}
\begin{itemize}[itemsep=0pt]
\item {\bf All India Rank - 190} in IIT-JEE, 2006 among more than 3,00,000 others.
\item Ranked among the {\bf Top 1\%} in the National Physics Olympiad(NSEP), 2006.
\item {\bf NTSE Scholar} : National Talent Search Examination 2004. 1000 students are selected for scholarship based on their and scientific and mental aptitude.
\end{itemize}


%%%%% EXTRA-CURRICULAR ACTIVITIES
%
%
\sectionhead{Extra-Curricular Activities}
\begin{itemize}[itemsep=0pt]
	\item \textbf{Shiksha Sopan summer camp `08} - Conducted classes on Physical Chemistry in an education camp for school students from nearby villages.
	\item \textbf{Web Coordinator}, Electrical Engineering Association, IIT Kanpur [2008-09] - Designed website and  conducted lectures on web development.
	\item \textbf{Student Guide}, Counselling Service, IIT Kanpur. [July-November, 2007]
	\item \textbf{Badminton} - Runners up in Udghosh '08, the annual sports festival of IIT Kanpur. 
	\item \textbf{Cycling} - I am daily commuter, and go for trail riding on weekends.
	\item \textbf{Online presence} - I like writing and sharing my learnings with others. I write a blog, \url{http://varuaa.blogspot.com/}. Recently while studying Graph Theory, I have started maintaining a webpage where I upload my notes - \url{http://varun729.github.com/graphtheory/}.
\end{itemize}


%%%%% FOOTER
%
%
%\sectionhead{Contact Details}
%A4, Vinayaka Apartments\\
%5th Cross, Ganesha Block, Dinnur Main Rd\\
%R.T.Nagar, Bangalore-560024\\
%INDIA\\
%Phone : +91-9620660970\\
%Email : varun729@gmail.com


%%%%% END OF DOCUMENT
%
%
\end{document}
